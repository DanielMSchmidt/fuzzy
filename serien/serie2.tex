\documentclass[12pt, a4paper]{article}
\usepackage{url,graphicx,tabularx,array,geometry}
\usepackage[utf8]{inputenc}
\usepackage[ngerman]{babel}
\usepackage{paralist}
\usepackage{latexsym}
\usepackage{fancyhdr}
\usepackage{siunitx}
\usepackage{graphicx}
\usepackage{float}
\usepackage{color}
\usepackage[utf8]{inputenc}

\pagestyle{fancy}

\usepackage{amsmath}
\usepackage{amsfonts}
\usepackage{amssymb}

\setlength{\parskip}{1ex} %--skip lines between paragraphs
\setlength{\parindent}{0pt} %--don't indent paragraphs

%-- Commands for header
\newcommand{\headerline}{\begin{tabularx}{\textwidth}{X>{\raggedleft}X}\hline\\\end{tabularx}\\[-0.5cm]}
\newcommand{\headerleftright}[2]{\begin{tabularx}{\textwidth}{X>{\raggedleft}X}#1%
& #2\\\end{tabularx}\\[-0.5cm]}
%\linespread{2} %-- Uncomment for Double Space

\usepackage{listings}
\usepackage{color}

\definecolor{dkgreen}{rgb}{0,0.6,0}
\definecolor{gray}{rgb}{0.5,0.5,0.5}
\definecolor{mauve}{rgb}{0.58,0,0.82}

\lstset{frame=tb,
  language=Java,
  aboveskip=3mm,
  belowskip=3mm,
  showstringspaces=false,
  columns=flexible,
  basicstyle={\small\ttfamily},
  numbers=none,
  numberstyle=\tiny\color{gray},
  keywordstyle=\color{blue},
  commentstyle=\color{dkgreen},
  stringstyle=\color{mauve},
  breaklines=true,
  breakatwhitespace=true
  tabsize=3
}

\begin{document}
\renewcommand{\headrulewidth}{0pt}
\fancyhf{}
\fancyhead[L]{
\headerleftright{\textbf{Fuzzy - Serie 2}}{David Elvers, Daniel Schmidt}}
\fancyfoot[C]{\thepage}

\section*{Aufgabe 5}
$\mu_{R_1} = \{ (X_1, 1), (X_2, 1), (X_3, 0,7), (X_4, 1) \}$ \\
$\mu_{R_2} = \{ (Y_1, 1), (Y_2, 0,7), (Y_3, 1), (Y_4, 1), (Y_5, 0,9) \}$

\section*{Aufgabe 6}
a) max - min - Komposition \\
 \begin{tabular}{c|ccc}
 & $Z_1$ & $Z_2$ & $Z_3$ \\
\hline \\
$X_1$ & 0 & 0 & 0 \\
$X_2$ & 0 & 0 & 0
 \end{tabular}
 \\
 \\
 b) max - Produkt - Komposition \\
  \begin{tabular}{c|ccc}
 & $Z_1$ & $Z_2$ & $Z_3$ \\
\hline \\
$X_1$ & 0,8 & 0,72 & 0,8 \\
$X_2$ & 1 & 0,9 & 1
 \end{tabular}
 \\
 \\
 c) max - Durchschnitts - Komposition \\
 \begin{tabular}{c|ccc}
 & $Z_1$ & $Z_2$ & $Z_3$ \\
\hline \\
$X_1$ & 0,9 & 0,85 & 0,9 \\
$X_2$ & 1 & 0,95 & 1
 \end{tabular}
\end{document}

