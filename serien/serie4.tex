\documentclass[12pt, a4paper]{article}
\usepackage{url,graphicx,tabularx,array,geometry}
\usepackage[utf8]{inputenc}
\usepackage[ngerman]{babel}
\usepackage{paralist}
\usepackage{latexsym}
\usepackage{fancyhdr}
\usepackage{siunitx}
\usepackage{graphicx}
\usepackage{float}
\usepackage{color}
\usepackage[utf8]{inputenc}

\pagestyle{fancy}

\usepackage{amsmath}
\usepackage{amsfonts}
\usepackage{amssymb}
\usepackage{pgfplots}

\setlength{\parskip}{1ex} %--skip lines between paragraphs
\setlength{\parindent}{0pt} %--don't indent paragraphs

%-- Commands for header
\newcommand{\headerline}{\begin{tabularx}{\textwidth}{X>{\raggedleft}X}\hline\\\end{tabularx}\\[-0.5cm]}
\newcommand{\headerleftright}[2]{\begin{tabularx}{\textwidth}{X>{\raggedleft}X}#1%
& #2\\\end{tabularx}\\[-0.5cm]}
%\linespread{2} %-- Uncomment for Double Space

\usepackage{listings}
\usepackage{color}

\definecolor{dkgreen}{rgb}{0,0.6,0}
\definecolor{gray}{rgb}{0.5,0.5,0.5}
\definecolor{mauve}{rgb}{0.58,0,0.82}

\lstset{frame=tb,
  language=Java,
  aboveskip=3mm,
  belowskip=3mm,
  showstringspaces=false,
  columns=flexible,
  basicstyle={\small\ttfamily},
  numbers=none,
  numberstyle=\tiny\color{gray},
  keywordstyle=\color{blue},
  commentstyle=\color{dkgreen},
  stringstyle=\color{mauve},
  breaklines=true,
  breakatwhitespace=true
  tabsize=3
}

\begin{document}
\renewcommand{\headrulewidth}{0pt}
\fancyhf{}
\fancyhead[L]{
\headerleftright{\textbf{Fuzzy - Serie 4}}{David Elvers, Daniel Schmidt}}
\fancyfoot[C]{\thepage}

\section*{Aufgabe 12}

$\mu_{A^{-1}}$ ist definiert durch: 

\begin{equation}
\mu_{A^{-1}}(x) = \left\{
 \begin{array}{lll}
 	\frac{1}{x+2} & \text{falls } -2 \le x \le -1 \\
	\frac{2}{-x+1} & \text{falls } -1 \le x \le 1 \\
	0 & \text{ sonst }
\end{array}
\right.	
\end{equation}

Graphisch sehen beide Fuzzy-Zahlen so aus

\begin{tikzpicture}
	\begin{axis}[width=500pt,axis x line=middle, axis y line=center, tick align=outside]
	\addplot coordinates {
	(-2,    0)
	(-1.5, 0.5)
	(-1,    1)
	(-0.5, 0.75)
	(0,      0.5)
	(0.5,   0.25)
	(1,      0)
	(1.5,   0)
	};
	\addlegendentry{A}
	
	\addplot coordinates {
	(-1.75,4)
	(-1.5,  2)
	(-1,     1)
	(-0.5,  1.33333)
	(0,      2)
	(0.5,   4)
	(0.75, 8)
	};
	\addlegendentry{$A^{-1}$}
	\end{axis}
\end{tikzpicture}

Da bei $\mu_{A^{-1}}$ durch 0 geteilt werden muss ist diese keine Funktion, dementsprechend kann es keine Zuordnungsfunktion zu $A^{-1}$ sein, wodurch dies keine Fuzzy-Zahl ist.

\section*{Aufgabe 13}



\end{document}

