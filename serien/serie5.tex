\documentclass[12pt, a4paper]{article}
\usepackage{url,graphicx,tabularx,array,geometry}
\usepackage[utf8]{inputenc}
\usepackage[ngerman]{babel}
\usepackage{paralist}
\usepackage{latexsym}
\usepackage{fancyhdr}
\usepackage{siunitx}
\usepackage{graphicx}
\usepackage{float}
\usepackage{color}
\usepackage[utf8]{inputenc}

\pagestyle{fancy}

\usepackage{amsmath}
\usepackage{amsfonts}
\usepackage{amssymb}
\usepackage{pgfplots}

\setlength{\parskip}{1ex} %--skip lines between paragraphs
\setlength{\parindent}{0pt} %--don't indent paragraphs

%-- Commands for header
\newcommand{\headerline}{\begin{tabularx}{\textwidth}{X>{\raggedleft}X}\hline\\\end{tabularx}\\[-0.5cm]}
\newcommand{\headerleftright}[2]{\begin{tabularx}{\textwidth}{X>{\raggedleft}X}#1%
& #2\\\end{tabularx}\\[-0.5cm]}
%\linespread{2} %-- Uncomment for Double Space

\usepackage{listings}
\usepackage{color}

\definecolor{dkgreen}{rgb}{0,0.6,0}
\definecolor{gray}{rgb}{0.5,0.5,0.5}
\definecolor{mauve}{rgb}{0.58,0,0.82}

\lstset{frame=tb,
  language=Java,
  aboveskip=3mm,
  belowskip=3mm,
  showstringspaces=false,
  columns=flexible,
  basicstyle={\small\ttfamily},
  numbers=none,
  numberstyle=\tiny\color{gray},
  keywordstyle=\color{blue},
  commentstyle=\color{dkgreen},
  stringstyle=\color{mauve},
  breaklines=true,
  breakatwhitespace=true
  tabsize=3
}

\begin{document}
\renewcommand{\headrulewidth}{0pt}
\fancyhf{}
\fancyhead[L]{
\headerleftright{\textbf{Fuzzy}}{David Elvers, Daniel Schmidt}}
\fancyfoot[C]{\thepage}

\section*{Aufgabe 14}

$A+B = (9,4,3)_{LR}$. Die entsprechende Zugehörigkeitsfunktion $\mu_{A+B}$ ist definiert durch:

\begin{equation}
\mu_{A+B}(x) = \left\{
 \begin{array}{lll}
 	max(0, \frac{9-x}{4})& \text{falls } x \le 9 \\
	\frac{1}{1+ \frac{x-9}{3}} & \text{falls } x \geq 9 \\
\end{array}
\right.	
\end{equation}

$-A = (-4, 2, 1)_{LR}$. Die entsprechende Zugehörigkeitsfunktion $\mu_{-A}$ ist definiert durch:

\begin{equation}
\mu_{A+B}(x) = \left\{
 \begin{array}{lll}
 	max(0, 1 - \frac{-4 -x}{2})& \text{falls } x \le -4 \\
	\frac{1}{x+5} & \text{falls } x \geq -4 \\
\end{array}
\right.	
\end{equation}

Graphisch sehen beide Fuzzy-Zahlen so aus:

\begin{tikzpicture}
	\begin{axis}[width=500pt,axis x line=middle, axis y line=center, tick align=outside]
	\addplot coordinates {
	(-5, 3.5)
	(0, 2.25)
	(1, 2)
	(9, 0)
	(9, 1)	
	(10, 0.75)
	(11, 0.6)
	};
	\addlegendentry{A+B}
	
	\addplot coordinates {
	(-5, 5.5)
	(-4, 5)
	(-4, 1)
	(-3, 0.5)
	(11, 0.0625)
	};
	\addlegendentry{-A}
	\end{axis}
\end{tikzpicture}

\section*{Aufgabe 15}

\end{document}

